\chapter{Lectures}

\section{Lecture 6: 04.09.2025}

Now let's consider more complex BCs
\begin{align*}
    -\Delta \mathbf{u}                 & = \mathbf{f} \quad \text{in } \Omega   \\
    \mathbf{u}                         & = \mathbf{g} \quad \text{on } \Gamma_D \\
    \nabla \mathbf{u} \cdot \textbf{n} & = l \quad \text{on } \Gamma_N
\end{align*}
where $\Gamma_D \cup \Gamma_N = \partial \Omega$ and $\Gamma_D \cap \Gamma_N = \emptyset$.

We again test with an arbitrary function $\mathbf{v}$:
\begin{align*}
    \langle \mathbf{f}, \mathbf{v} \rangle & = \langle -\Delta \mathbf{u}, \mathbf{v} \rangle = \langle \nabla \mathbf{u}, \nabla \mathbf{v} \rangle + \int_{\partial \Omega} \nabla \mathbf{u} \cdot \textbf{n} \, \mathbf{v} \, ds
\end{align*}
As we have multiple BCs, on different subdomains we can consider them independently.
\begin{align*}
    \int_{\partial \Omega} \nabla \mathbf{u} \cdot \textbf{n} \, \mathbf{v} \, ds & = \int_{\Gamma_D} \nabla \mathbf{u} \cdot \textbf{n} \, \mathbf{v} \, ds + \int_{\Gamma_N} \nabla \mathbf{u} \cdot \textbf{n} \, \mathbf{v} \, ds = 0 + \int_{\Gamma_N} l \, \mathbf{v} \, ds
\end{align*}

\begin{itemize}
    \item \textbf{Neumann:} The natural BC becomes:
          \[
              \int_{\Gamma_N} \nabla \mathbf{u} \cdot \textbf{n} \, \mathbf{v} \, ds = \int_{\Gamma_N} l \, \mathbf{v} \, ds
          \]
          and remains in our formulation.
    \item \textbf{Dirichlet:} By specifying $\mathbf{u} \in \mathcal{H}^1_{\Gamma_D}(\Omega)$ where $\mathcal{H}^1_{\Gamma_D}(\Omega) = \{ \mathbf{v} \in \mathcal{H}^1(\Omega) : \mathbf{v}\vert_{\Gamma_D} = \mathbf{g} \}$ we obtain the weak form:
          \[
              \text{Find } \mathbf{u} \in \mathcal{H}^1_{\Gamma_D}(\Omega) \text{ s.t. } \langle \nabla \mathbf{u}, \nabla \mathbf{v} \rangle = \langle \mathbf{f}, \mathbf{v} \rangle + \int_{\Gamma_N} l \, \mathbf{v} \, ds \quad \forall \mathbf{v} \in \mathcal{H}^1_0(\Omega)
          \]
\end{itemize}

\textbf{Warning:} Trial and test spaces dont match! And $\mathcal{H}^1_{\Gamma_D}(\Omega) \nsubset \mathcal{H}^1(\Omega)$.
\textbf{Solution:} Lift the solution $\mathbf{u}$ so we solve a homogeneous problem.
Suppose we have operator $R_g \in \mathcal{H}^1(\Omega)$ s.t. $R_g\vert_{\Gamma_D} = \mathbf{g}$, then we set $\odot{\mathbf{u}} = \mathbf{u} - R_g$ and solve for $\odot{\mathbf{u}}$:
\[
    \text{Find } \odot{\mathbf{u}} \in \mathcal{H}^1_D(\Omega) \text{ s.t. } \langle \nabla \odot{\mathbf{u}}, \nabla \mathbf{v} \rangle = \langle \mathbf{f}, \mathbf{v} \rangle + \int_{\Gamma_N} l \, \mathbf{v} \, ds - \langle \nabla R_g, \nabla \mathbf{v} \rangle \quad \forall \mathbf{v} \in \mathcal{H}^1_D(\Omega)
\]
where
\[
    \mathcal{H}^1_D(\Omega) = \{ \mathbf{v} \in \mathcal{H}^1(\Omega) : \mathbf{v}\vert_{\Gamma_D} = 0 \}
\]
Thus the problem is \emph{symmetric} again!

\textit{Exercise:} More general elliptic problems:

Find the weak form for the PDE
\begin{align*}
    -\operatorname{div}{(\mu \nabla \mathbf{u})} + \sigma \mathbf{u} & = \mathbf{f} \quad \text{in } \Omega   \\
    \mathbf{u}                                                       & = \mathbf{g} \quad \text{on } \Gamma_D \\
    \mu \nabla \mathbf{u} \cdot \textbf{n}                           & = l \quad \text{on } \Gamma_N
\end{align*}

\subsection*{Functional analysis recap (also see Q2)}
A functional $F$ on $V$ is an operator mapping $F: V \to \mathbb{R}$.
The functional we have in mind is :
\[F = \langle \mathbf{f}, \mathbf{v} \rangle\]
A linear functional is said to be bounded if $\exists C > 0$ s.t.
\[|F(\mathbf{v})| \leq C \|\mathbf{v}\|_V \quad \forall \mathbf{v} \in V\]
If a functional is linear and bounded on some Banach space, then it is continuous.

\textit{Banach space: }
\begin{itemize}
    \item A vector space $V$ over $\mathbb{R}$ or $\mathbb{C}$
    \item Equipped with a norm $\|\cdot\|_V$
    \item Complete w.r.t. the metric induced by the norm
\end{itemize}

A bilinear form $a: V \times V \to \mathbb{R}$ is continuous if $\exists M > 0$ s.t.
\[|a(\mathbf{u}, \mathbf{v})| \leq M \|\mathbf{u}\|_V \|\mathbf{v}\|_V \quad \forall \mathbf{u}, \mathbf{v} \in V\]
and coercive if $\exists \alpha > 0$ s.t.
\[a(\mathbf{v}, \mathbf{v}) \geq \alpha \|\mathbf{v}\|_V^2 \quad \forall \mathbf{v} \in V\]

\begin{theorem}{Lax-Milgram}{}
    Let $V$ be a Hilbert space, $a(\cdot, \cdot): V \times V \to \mathbb{R}$ a continuous and coercive bilinear form, and $F: V \to \mathbb{R}$ a continuous linear functional. Then there exists a unique solution to the problem:
    \[
        \text{Find } \mathbf{u} \in V \text{ s.t. }
        a(\mathbf{u}, \mathbf{v}) = F(\mathbf{v}) \quad \forall \mathbf{v} \in V
    \]
    Moreover, $\|\mathbf{u}\|_V \leq \frac{1}{\alpha} \|F\|_{V'}$ where $\|F\|_{V'} = \sup_{\mathbf{v} \in V \setminus \{0\}} \frac{|F(\mathbf{v})|}{\|\mathbf{v}\|_V}$.
\end{theorem}
\begin{itemize}
    \item Based on \emph{Riesz representation theorem} and the \emph{Banach fixed point theorem}.
    \item To prove unique solutions exists we need to satisfy the conditions on the linear $F$ and bilinear form $a(\cdot, \cdot)$.
\end{itemize}

\begin{example}{}{}
    Consider seeking $u \in H^1_0([0, 1])$ s.t.

    \begin{align*}
        \int_0^1 u'(x) v'(x) \, dx & = \int_0^1 \sin(2\pi x) v(x) \, dx \quad \forall v \in H^1_0([0, 1]) \\
        a(u, v)                    & = \int_0^1 u'(x) v'(x) \, dx                                         \\
        F(v)                       & = \int_0^1 \sin(2\pi x) v(x) \, dx
    \end{align*}
    Clearly
    \begin{align*}
        |F(v)|        & \leq \|\sin(2\pi x)\|_{L^2} \|v\|_{L^2} \leq \|v\|_{L^2} \leq \|v\|_{H^1_0} \\
        \|v\|_{H^1}^2 & = \|v\|_{L^2}^2 + \|\nabla v\|_{L^2}^2 \geq \|v\|_{L^2}^2
    \end{align*}
    so $F$ is bounded and linear. Also
    \begin{align*}
        |a(u, v)| & = |\langle u, v \rangle_{H^1_0}| \leq \|u\|_{H^1_0} \|v\|_{H^1_0} \quad \text{(Cauchy-Schwarz)}
    \end{align*}
    thus continuous. For coercivity we must use the Poincaré inequality:
    \[\|v\|_{L^2} \leq C \|\nabla v\|_{L^2} \quad \forall v \in H^1_0(\Omega)\]
    so
    \begin{align*}
        a(v, v) & = \|\nabla v\|_{L^2}^2 \geq \frac{1}{C^2} \|v\|_{L^2}^2 \geq \frac{1}{C^2 + 1} \|v\|_{H^1_0}^2
    \end{align*}
    and we have coercivity.
    Thus by Lax-Milgram a unique solution exists.
\end{example}

\subsection*{Elliptic finite elements in 1D}
We have discussed the infinitely dimensional case, now we want to move into the finite dimensional case.

\begin{definition}{Ritz-Galerkin approximation}{}
    Let $a: \mathbb{V} \times \mathbb{V} \to \mathbb{R}$ be a bilinear form, and $\mathbb{V}_h \subset \mathbb{V}$ a finite dimensional subspace.
    Consider the weak form restricted to $\mathbb{V}_h$:
    \[\text{Find } \mathbf{u}_h \in \mathbb{V}_h \quad \text{s.t.} \quad a(\mathbf{u}_h, \mathbf{v}) = F(\mathbf{v}) \quad \forall \mathbf{v} \in \mathbb{V}_h\]
    This is called a Ritz-Galerkin approximation of the weak solution $\mathbf{u} \in \mathbb{V}$.
\end{definition}

\begin{example}{Polynomial subspace for the 1D problem}{}
    Consider a (naive) approx. of the form:
    \[
        \hat{u}(x) = a_0 + a_1 x + a_2 x^2 + a_3 x^3
    \]
    to our 1D problem:
    \begin{align*}
        -u''(x) & = f(x) \quad x \in (0, 1) \\
        u(0)    & = 0                       \\
        u'(1)   & = 0
    \end{align*}
    with the weak form:
    \[\text{Find } u \in H^1_0(0, 1) \quad \text{s.t.} \quad a(u, v) = F(v) \quad \forall v \in H^1_0(0, 1)\]
    where
    \begin{align*}
        a(u, v) & = \int_0^1 u'(x) v'(x) \, dx \\
        F(v)    & = \int_0^1 f(x) v(x) \, dx
    \end{align*}
    We observe that $\hat{u}(0) = a_0 = 0$ so we can rewrite:
    \[
        \hat{u}(x) = a_1 x + a_2 x^2 + a_3 x^3
    \]
    The monomials form a polynomial basis of a subspace of $\mathbb{V}_h \subset \mathbb{V}$ where we consider an approx. over a single element. The coefficients of $\hat{u}$ are determined by the constraints:
    \begin{align*}
        \langle \hat{u}', 1 \rangle    & = \langle f, x \rangle   \\
        \langle \hat{u}', 2x \rangle   & = \langle f, x^2 \rangle \\
        \langle \hat{u}', 3x^2 \rangle & = \langle f, x^3 \rangle
    \end{align*}
    This is because we want to satisfy the weak form for all $v \in \mathbb{V}_h$ and we have three basis functions.
    This gives us a linear system of equations for the coefficients $a_1, a_2, a_3$.
    \begin{align*}
        \begin{bmatrix}
            1 & 1           & 1           \\
            1 & \frac{4}{3} & \frac{3}{2} \\
            1 & \frac{3}{2} & \frac{9}{5}
        \end{bmatrix}
        \begin{bmatrix}
            a_1 \\
            a_2 \\
            a_3
        \end{bmatrix}
         & =
        \begin{bmatrix}
            \langle f, x \rangle   \\
            \langle f, x^2 \rangle \\
            \langle f, x^3 \rangle
        \end{bmatrix}
        = \begin{bmatrix}
              1                  \\
              -\frac{4}{\pi} + 2 \\
              \frac{\pi^2}{4} \left(-\frac{96}{\pi^4} + \frac{12}{\pi^2} \right)
          \end{bmatrix}
    \end{align*}
    which we can solve for:
    \[
        \hat{u}(x) \approx 1.6x - 0.16x^2 - 0.44x^3
    \]
    which remains close to the true solution
    \[
        u(x) = \sin(\frac{\pi x}{2}) \text{ over } [0, 1]
    \]
\end{example}

\textbf{Questions:}
\begin{itemize}
    \item Does a discrete solution always exist?
    \item Is it unique?
    \item How accurate is it?
\end{itemize}

\begin{theorem}{1D existence and uniqueness for $-u''(x) = f(x)$ with $u(0) = u'(1) = 0$}{}
    If $f \in L^2([0, 1])$ then $\exists! u_h$ to the Ritz-Galerkin approx. when $\mathbb{V}_h \subset \mathbb{V}$ is finite dimensional.
\end{theorem}
Since $\mathbb{V}_h$ is finite dimensional it has a finite dimensional basis:
\[\mathbb{V}_h = \text{span}\{\phi_1, \phi_2, \ldots, \phi_N\}\]
So, for any $u_h \in \mathbb{V}_h$ where:
\begin{align*}
    u_h(x)         & = \sum_{i=1}^n \alpha_i \phi_i(x) \\
    \symbf{\alpha} & = \{\alpha_i\}_{i=1}^n
\end{align*}
The Ritz-Galerkin approx. seeks $\symbf{\alpha}$ s.t.
\begin{align*}
    a(u_h, v)                                     & = F(v) \quad \forall v \in \mathbb{V}_h               \\
    a\left(\sum_{i=1}^n \alpha_i \phi_i, v\right) & = \langle f, \phi_j \rangle \quad j = 1, 2, \ldots, n
\end{align*}
Since any $v \in \mathbb{V}_h$ is also a linear combination of basis functions this is equivalent to seeking $\symbf{\alpha}$ s.t.
\begin{align*}
    a\left(\sum_{i=1}^n \alpha_i \phi_i, \phi_j\right) & = \langle f, \phi_j \rangle \quad j = 1, 2, \ldots, n
\end{align*}
where the rhs is finite as $f \in L^2([0, 1])$.

Through bilinearity:
\begin{align*}
    \sum_{i=1}^n \alpha_i a(\phi_i, \phi_j) & = \langle f, \phi_j \rangle, \quad j = 1, 2, \ldots, n
\end{align*}
Defining the matrix $A$ by $A_{ij} = a(\phi_i, \phi_j)$ and the vector $\mathbf{b}$ by $b_j = \langle f, \phi_j \rangle$, our approximation is equivalent to solving the linear system:
\[A \symbf{\alpha} = \mathbf{b}\]
Existence and uniqueness is equivalent to solving $A \symbf{\alpha} = \mathbf{b}$.
Assume $A$ is singular, i.e.
\[
    \exists \symbf{\beta} \neq 0 \text{ s.t. } A \symbf{\beta} = 0
\]
and
\[
    \symbf{\beta}^\top A \symbf{\beta} = 0
\]
We now define:
\[
    \tilde{v}(x) = \sum_{i=1}^n \beta_i \phi_i(x) \in \mathbb{V}_h
\]
We see that:
\begin{align*}
    \symbf{\beta}^\top A \symbf{\beta} & = \sum_{i=1}^n \sum_{j=1}^n \beta_i a(\phi_i, \phi_j) \beta_j \\
                                       & = a(\sum_{i=1}^n \beta_i \phi_i, \sum_{j=1}^n \beta_j \phi_j) \\
                                       & = a(\tilde{v}, \tilde{v})= \int_0^1 \tilde{v}'(x)^2 \, dx = 0
\end{align*}
Now, if:
\begin{align*}
    \symbf{\beta}^\top A \symbf{\beta} = 0 \implies \tilde{v}'(x) = 0 \, \forall x, implies \tilde{v}(x) = C \\
\end{align*}
As $\tilde{v} \in \mathbb{V}_h \subset \mathbb{V} = H^1_0([0, 1])$ it must satisfy the BC $\tilde{v}(0) = 0$ so $C = 0$ and $\symbf{\beta} = 0$ which is a contradiction.
As $A$ must be non-singular, $A \symbf{\alpha} = \mathbf{b}$ has a unique solution $\symbf{\alpha}$ and thus a unique solution.

\section{Lecture 7: 10.09.2025}
\begin{lemma}{Galerkin Orthogonality}{}
    Let $u \in V$ be the solution to the weak form and $u_h \in V_h \subset V$ the Ritz-Galerkin approximation.

    \[
        a(u - u_h, v) = 0 \quad \forall v \in V_h
    \]
\end{lemma}
Recall the weak form:
\[
    a(u, v) = F(v) \quad \forall v \in V
\]
and the Ritz-Galerkin approx.:
\[
    a(u_h, v) = F(v) \quad \forall v \in V_h
\]
As $V_h \subset V$ through restricting the weak form and:
\[
    a(u - u_h, v) = \langle f, v \rangle - \langle f, v \rangle = 0 \quad \forall v \in V_h
\]
\begin{lemma}{Cea's Lemma}{ceas-lemma}
    If $u \in V$ solves the weak form and $u_h \in V_h$ the Ritz-Galerkin approximation, then:
    \[
        \|u - u_h\|_V = \min_{v \in V_h} \|u - v\|_V
    \]
\end{lemma}

Recall that:
\[
    \|u - u_h\|_V^2 = a(u - u_h, u - u_h)
\]
For any $v \in V_h$ we have:
\begin{align*}
    a(u - u_h, u - u_h) & = a(u - u_h, u - v) + a(u - u_h, v - u_h)
\end{align*}

Noting that $v - u_h \in V_h$ we use Galerkin orthogonality to eliminate the second term:
\begin{align*}
    \|u - u_h\|_V^2 & = a(u - u_h, u - v) \quad \forall v \in V_h
\end{align*}
As $a(\cdot, \cdot)$ is an inner product (continuous and coercive) we can use the Cauchy-Schwarz inequality:
\begin{align*}
    \|u - u_h\|_V^2 & \leq \|u - u_h\|_V \|u - v\|_V \quad \forall v \in V_h
\end{align*}
as the statement holds for all $v \in V_h$ it also holds for the minimiser:
\[\|u - u_h\|_V^2 \leq \|u - u_h\|_V \min_{v \in V_h} \|u - v\|_V \qed\]

Before proving error estimates in 1D we need some assumptions and results:
\begin{itemize}
    \item \textbf{Approximation assumption:} Given $V_h \subset V$, assume for $V_h$ there $\exists \varepsilon > 0$ s.t. $\forall u \in C^2([0, 1]) \cap V$ where:
          \[
              \min_{v \in V_h} \|u - v\|_V \leq \varepsilon \|u''\|_{L^2}
          \]
    \item \textbf{Theorem: Aubin-Nitsche duality argument:}
          Let $f \in L^2([0, 1])$, $u \in V$ solves the weak form and $u_h \in V$ the Ritz-Galerkin approx. If the approx. assumption holds then:
          \[\|u - u_h\|_{L^2} \leq \varepsilon \|u - u_h\|_V\]
          \emph{Proof:}
          Let $w \in C^2([0, 1]) \cap V$ solve the dual problem:
          \begin{align*}
              -w''  & = u - u_h, \quad x \in (0, 1) \\
              w(0)  & = 0                           \\
              w'(1) & = 0
          \end{align*}
          Then $w$ solves the weak form:
          \[a(w, v) = \langle u - u_h, v \rangle \quad \forall v \in V\]
          Since $u - u_h \in V$ we have:
          \begin{align*}
              \|u - u_h\|_{L^2}^2 & = \langle u - u_h, u - u_h \rangle = a(w, u - u_h)                                       \\
                                  & = a(w, u - u_h) - a(u - u_h, v) \quad \forall v \in V_h \text{ (Galerkin orthogonality)}
          \end{align*}
          Through Galerkin orthogonality. Through Cauchy-Schwarz (CS):
          \[
              \|u - u_h\|_{L^2}^2 \leq \|u - u_h\|_V \|w - v\|_V \quad \forall v \in V_h
          \]
          Choosing $v$ to be the minimiser we can use the approx. assumption:
          \[
              \|u - u_h\|_{L^2}^2 \leq \varepsilon \|u - u_h\|_V \|w''\|_{L^2} = \varepsilon \|u - u_h\|_V \|u - u_h\|_{L^2}
          \]
          by construction of the dual problem, thus:
          \[
              \|u - u_h\|_{L^2} \leq \varepsilon \|u - u_h\|_V \qed
          \]
\end{itemize}

\begin{corollary}{}{}
    Let the assumptions of \emph{Aubin-Nitsche} hold. If $f \in C^0([0, 1])$ and $u \in C^2([0, 1])$, then:
    \[
        \|u - u_h\|_{L^2} \leq \varepsilon \|u - u_h\|_V \leq \varepsilon^2 \|f\|_{L^2}
    \]
\end{corollary}
From \ref{lem:ceas-lemma} we have:
\[\|u - u_h\|_V = \min_{v \in V_h} \|u - v\|_V\]
and the approx. assumption:
\[\min_{v \in V_h} \|u - v\|_V \leq \varepsilon \|u''\|_{L^2}\]
we see that:
\[\varepsilon \|u - u_h\|_V \leq \varepsilon \min_{v \in V_h} \|u - v\|_V \leq \varepsilon^2 \|u''\|_{L^2}.\]
Since $u \in C^2([0, 1])$ it is a strong solution of the PDE and $\|u''\|_{L^2} = \|f\|_{L^2}$ allowing us to conclude:
\[\|u - u_h\|_{L^2} \leq \varepsilon \|u - u_h\|_V \leq \varepsilon^2 \|f\|_{L^2} \qed\]

\subsection{Meshes}
\begin{definition}{Mesh}{mesh}
    A mesh on $[0, 1]$ is given by a set of nodes satisfying:
    \[0 = x_0 < x_1 < x_2 < \ldots < x_n= 1\]
    which divide $[0, 1]$ into $n$ elements:
    \[I_i = [x_i, x_{i+1}], \quad i = 0, 1, \ldots, n-1\]
    The mesh spacing is defined by $h_i = x_{i+1} - x_i$.
\end{definition}

\begin{definition}{Finite Element Space}{fe-space}
    For $k \geq 1$ we define:
    \begin{align*}
        V_h^k &= \{v \in C^0([0, 1]) : v(x) \text{ is polynomial of degree }\leq k \text{ on each element } I_i \text{ and } v(0) = 0\} \\
              &= \{v \in C^0([0, 1]) : v(x)\vert_{I_i} \in \mathbb{P}_k, \, i = 0, 1, \ldots, n-1, \, v(0) = 0\}
    \end{align*}
\end{definition}

\section{Lecture 8: 11.09.2025}

Recall if $k=1$ we may use nodal basis functions:
\[
\phi_i(x)
= \begin{cases}
    \frac{x - x_{i-1}}{h_{i-1}}, & x \in [x_{i-1}, x_i] \\
    \frac{x_{i+1} - x}{h_i},     & x \in [x_i, x_{i+1}] \\
    0,                           & \text{otherwise}
\end{cases}
\quad i = 1, 2, \ldots, n-1
\]
where $\phi_i(x_j) = \delta_{ij}$ and $\phi_i(0) = 0$.